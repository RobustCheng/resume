% !TEX TS-program = xelatex
% !TEX encoding = UTF-8 Unicode
% !Mode:: "TeX:UTF-8"

\documentclass{resume}
\usepackage{graphicx}
\usepackage{tabu}
\usepackage{tabularx}
\usepackage{multirow}
\usepackage{progressbar}
\usepackage{zh_CN-Adobefonts_external} % Simplified Chinese Support using external fonts (./fonts/zh_CN-Adobe/)
\usepackage{tikz}
% \usepackage{NotoSansSC_external}
% \usepackage{NotoSerifCJKsc_external}
% \usepackage{zh_CN-Adobefonts_internal} % Simplified Chinese Support using system fonts
\usepackage{linespacing_fix} % disable extra space before next section
\usepackage{cite}

%\newcommand{\hlink}[1]{\href{#1}{#1}}

\begin{document}
\pagenumbering{gobble} % suppress displaying page number

%\medskip\noindent
\begin{minipage}{0.7\textwidth}
  \Large{
    \begin{tabu}  { l }
      \scshape{彭 \quad  程} \\
      \email{pengcheng@live.ca} \\
      \phone{(+86) 13070157185} \\
    \end{tabu}
  }
\end{minipage}
\begin{minipage}{0.3\textwidth}
  \raggedleft
  \includegraphics[height=30mm]{me}
\end{minipage}

\section{ 教育背景}
\datedsubsection{\textbf{北京航空航天大学}, 北京}{2017年9月 -- 2020年1月(预计)}
\textit{在读硕士研究生}\quad {计算机学院} \quad{研究生专业学位硕士研究生考试总成绩第一录取}
\datedsubsection{\textbf{北京航空航天大学}, 北京}{2013年9月 -- 2017年6月}
\textit{学士}\quad  {电子信息工程学院}\quad { 北斗实验班}

\section{ 实习/项目经历}

\datedsubsection{\textbf{大规模图像的分级快速检索}}{2017年12月 -- 至今}
\role{CNN 深度哈希}{研究生课题}
%\begin{onehalfspacing}
\begin{itemize}[topsep = 0 pt, partopsep = 0pt]
  \item 多级检索的快速深度哈希索引CBIR算法,使用深度学习网络生成二进制哈希码
  \item 百万量级图像库,将检索时间从目前的几秒缩减至100毫秒以内
\end{itemize}
%\end{onehalfspacing}


\datedsubsection{\textbf{某部队遥感影像检测项目}}{2018年5月 -- 2019年1月}
\role{Mask R-CNN}{实验室项目}
%\begin{onehalfspacing}
\begin{itemize}[topsep = 0 pt, partopsep = 0pt]
  \item 对可见光或融合影像,检测地面建筑物,并对相同地点不同时间的影像做变化检测
  \item 用 Mask R-CNN对建筑物做检测。使用PCA降维、K-means算法与形态学算法做变化检测
  \item 对小尺寸目标用超分辨率方法提高准确率
\end{itemize}
%\end{onehalfspacing}

\datedsubsection{\textbf{旷视科技数据平台建设}}{2017年7月 -- 2018年2月}
\role{Spark,Storm}{旷视CSG数据平台实习生}
\begin{itemize}[topsep = 0 pt, partopsep = 0pt]
  \item 参与旷视科技CSG(云服务部门)数据后台建设。提升平台随面对业务增长带来的大数据流量的应对能力与扩展能力
  \item 用Spark、Hive协助数据分析师分析数据,识别可疑用户
  \item Storm 实时流处理,消费Kafka数据,实现实时报表统计与预警
  \item 使用Airflow 任务调度,自动化周期性任务以节省人力
\end{itemize}

\datedsubsection{\textbf{基于虚拟现实技术的医学影像重建与交互系统}}{2016年6月 -- 2017年6月}
\role{Unity3D}{本科毕设,北航电子实验中心}
%\begin{onehalfspacing}
\begin{itemize}[topsep = 0 pt, partopsep = 0pt]
  \item 对CT、MRI等医疗影像进行三维重建,并在VR设备上现实与交互,可用与辅助诊断与教学
  \item 使用VTK库对CT、MRI影像做3D重建
  \item 模型导入Unity3D引擎,并在HTC Vive上显示与交互
  \item 自定义着色器对3D模型切片以观察内部结构
\end{itemize}
%\end{onehalfspacing}

\section{ 技能}
% increase linespacing [parsep=0.5ex]
\begin{itemize}[parsep=0.5ex]
  \item 编程语言:常用 C/C++、Python,了解 Java、Scala、Golang
  \item 工具: Linux、Git、Docker
\end{itemize}

\section{ 获奖情况}
\datedline{\textit{国家励志奖学金}}{2014 年 8 月}
\datedline{\textit{北京航空航天大学新生奖学金}}{2017 年 8 月}

\section{ 其他}
% increase linespacing [parsep=0.5ex]
\begin{itemize}[parsep=0.5ex]
  \item  英语 - CET-6 530
  \item  兴趣 - 游泳、桌游、羽毛球、无线电台呼号 BI1GPX
\end{itemize}

\end{document}
